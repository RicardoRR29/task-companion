\begin{resumo}[RESUMO]
\vspace*{-6mm}

A tecnologia de transcrição de fala tem evoluído bastante e se tornado mais utilizada devido o crescimento das plataformas de serviço em nuvem. Como existem diversos serviços que entregam essa funcionalidade, surge a dúvida de qual é a mais adequada para processar a língua portuguesa. Neste trabalho, foi realizada uma análise de duas ferramentas de mercado que realizam a transcrição de áudio, com o objetivo de determinar qual a taxa de assertividade de cada uma dessas ferramentas no tratamento de diferentes sotaques brasileiros. 
Para realizar essa comparação foi utilizada a base de dados Braccent que contém uma coleção de áudios que contemplam todas as regiões do Brasil, um conjunto de 1.648 áudios. Foram avaliados fatores regionais, fatores culturais e fatores naturais. Os fatores regionais foram os sete sotaques: nortista, baiano, fluminense, mineiro, carioca, nordestino e sulista; os fatores culturais foi a informação de grau de escolaridade: médio incompleto, médio completo, superior incompleto, superior completo, mestrado e doutorado; e os fatores naturais foi o gênero: masculino e feminino.  Foi realizado a transcrição desses áudios utilizando uma aplicação que fez requisições aos serviços do Google Cloud e Wit.ai. As métricas Levenshtein e Levenshtein Normalizado foram utilizadas para avaliar a taxa de acerto da transcrição da fala. 
Comparando os dados das duas APIs, é possível perceber que a Wit.ai apresenta melhores resultados que o Google Cloud. A média da métrica de Levenshtein Normalizado é de 0,96, enquanto o Google Cloud é de 0,89 considerando símbolos de pontuação. O Wit.ai acertou 81 áudios completos com pontuação. Ressalta-se que sem considerar os símbolos de pontuação, essa quantidade aumenta em 6,7 vezes, acertando 543 áudios. Na análise dos resultados por sotaque, o Google Cloud obteve melhores resultados para o sotaque baiano seguido do nordestino, e os  piores resultados foram para o sotaque carioca e não houve um único acerto para o sotaque nortista. Em todos os sotaques, o Wit.ai apresentou resultados melhores que o Google Cloud. O Wit.ai foi bem em todos os sotaques, apresentando um resultado um pouco pior para o sotaque carioca. 
Os resultados do Google Cloud são melhores para o gênero feminino que o do masculino. Nos resultados do  Wit.ai também, mas a diferença é bem pequena podendo ser  considerada equiparável. Quanto aos resultados por grau de escolaridade, é possível verificar que o nível de ensino não é um fator discriminador para as ferramentas de transcrição considerando a base de dados usada. Ao final, considera-se que o Wit.ai apresentou resultados melhores em todos os cenários, além de transcrever a os símbolos de pontuação.


Palavras-chave: Base de dados Braccent. Fala-para-texto. Google Cloud. Wit.ai. Reconhecimento Automático de Fala.


%A API Wit.ai apresentou melhores resultados em todas as comparações realizadas, além de que o Google Cloud não faz uso de pontuações. Através da metodologia proposta, conclui-se que a Wit.ai possui melhores resultados independente de qualquer característica do locutor (sexo, grau de ensino, região) para transcrições de áudios em PT-BR.

\end{resumo}
    