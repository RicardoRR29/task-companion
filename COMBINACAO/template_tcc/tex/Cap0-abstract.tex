
\begin{resumo}[ABSTRACT]
\begin{otherlanguage}{english}
\vspace*{-6mm}


Speech transcription technology has evolved a lot and has become more used due to the growth of cloud service platforms. As several services deliver this functionality, it raises the question of which is the most suitable to process the Portuguese language. In this work, an analysis of two market tools that perform audio transcription was carried out, aiming to determine the rate of assertiveness of each of these tools in treating different Brazillian accents. The Braccent database was used to perform this comparasion, which contains a collection of audios that covers all regions of Brazil, a subset of 1,648 audios. It was possible to evaluate regional factors, cultural factors, and natural factors. The regional factors were the seven accents: nortista, baiano, fluminense, mineiro, carioca, nordestino e sulista; the cultural factors were the information on the level of education: incomplete high school, complete high school, incomplete higher education, complete higher education, master's and doctorate; and the natural factors were the gender: male and female. These audios were transcribed using an application that made requests to Google Cloud and Wit.ai services. The Levenshtein and Levenshtein Normalized metrics were used to assess the rate of the correctness of speech transcription. Comparing the data from the two APIs, it is possible to see that Wit.ai presents better results than Google Cloud. The average of the Normalized Levenshtein metric is 0.96, while Google Cloud is 0.89 considering the punctuation. Wit.ai got 81 complete audios with punctuation right. It is noteworthy that without considering the punctuation, this amount increases by 6.7 times, reaching 543 audios. In the analysis of the results by accent, Google Cloud obtained better results for the baiano accent followed by the nordestino, and the worst results were for the carioca accent and there was not a single correct answer for the nortista accent. In all accents, Wit.ai showed better results than Google Cloud. Wit.ai did well in all accents, showing a slightly worse result for the carioca accent. Google Cloud results are better for women than for men. In the Wit.ai results too, but the difference is very small and can be considered comparable. As for the results by level of education, it is possible to verify that the level of education is not a discriminating factor for the transcription tools considering the database used. In the end, it is considered that Wit.ai presented better results in all scenarios, in addition to transcribing the punctuation. 



Keywords: Braccent database. Speech-to-text. Google Cloud. Wit.ai. Automatic Speech Recognition.


  \end{otherlanguage}
\end{resumo}
