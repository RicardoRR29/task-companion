\documentclass[12pt,openright,twoside,a4paper]{abntex2}
\usepackage[utf8]{inputenc}
\usepackage[T1]{fontenc}
\usepackage{graphicx}
\usepackage{lipsum}
\usepackage{indentfirst}
\usepackage{microtype}
\usepackage{hyperref}

\title{TACO -- Task Companion}
\author{Seu Nome}
\date{\today}

\begin{document}

% Capa
\imprimircapa

% Folha de rosto
\imprimirfolhaderosto

% Folha de aprovação
\begin{folhadeaprovacao}
  \assinatura{\textbf{Orientador(a)}}
  \assinatura{\textbf{Examinador(a) 1}}
  \assinatura{\textbf{Examinador(a) 2}}
\end{folhadeaprovacao}

% Agradecimentos
\chapter*{Agradecimentos}
\lipsum[1]

% Resumo
\begin{resumo}
O sistema TACO (Task Companion) \cite{taco_repo} consiste em uma aplicação web desenvolvida com React e TypeScript, organizada pelo bundler Vite. Seu propósito é auxiliar a criação e execução de fluxos de atividades interativas, permitindo que usuários montem sequências de passos do tipo texto, pergunta, mídia, componente customizado ou webhook. Cada fluxo é salvo localmente por meio da biblioteca Dexie, que abstrai o IndexedDB, garantindo persistência sem necessidade de servidor.

O editor de fluxos possibilita operações de criar, clonar, importar e exportar definições. Na importação, os identificadores de passos são regenerados para evitar conflitos. Um grafo de transições é gerado automaticamente para cada fluxo, servindo de base para análise de caminhos percorridos. O módulo de reprodução (Flow Player) inicia sessões de execução, registra passagem por cada passo e permite pausar ou retomar o processo. Caso o usuário complete o fluxo, incrementa-se o contador de conclusões e é gravado um log detalhado.

Todos os eventos relevantes são armazenados em coleções específicas, possibilitando estatísticas como taxa de conclusão e tempo médio por passo. Além disso, há suporte a componentes customizados, configuração de cores e logotipo da empresa, lista de auditoria e exportação de dados. A interface utiliza Tailwind CSS e Radix UI para oferecer experiência responsiva e acessível, funcionando também como PWA. Assim, o TACO se apresenta como uma solução completa e local para gerenciamento de procedimentos e acompanhamento de produtividade.

O modo offline garante acesso irrestrito aos fluxos, beneficiando organizações que atuam em locais sem conectividade. Por meio de técnicas de gamificação e feedback contínuo, o sistema incentiva a conclusão das etapas e facilita o aprendizado prático. Os relatórios analíticos permitem otimizar os processos com base em evidências reais, tornando o TACO uma ferramenta versátil para empresas de diferentes portes e segmentos.

\end{resumo}

% Abstract
\begin{otherlanguage*}{english}
\begin{abstract}
% (Insira aqui o abstract em ingl\^es resumindo os objetivos, metodologia, resultados e conclus\~oes do trabalho.)

\end{abstract}
\end{otherlanguage*}

% Listas
(A lista de figuras abaixo \e gerada automaticamente a partir dos ambientes \texttt{figure}.)
\listoffigures
(A lista de tabelas abaixo \e gerada automaticamente a partir dos ambientes \texttt{table}; remova esta linha se n\~ao utilizar tabelas.)
\listoftables
\chapter*{Lista de siglas}
(Inclua aqui todas as abrevia\c{c}\~oes utilizadas no texto e seus significados.)
\begin{description}
  \item[TACO] Task Companion
  \item[DB] Database
  \item[PWA] Progressive Web App
  \item[API] Application Programming Interface
\end{description}

% Sumário
\tableofcontents

% Desenvolvimento do trabalho
\chapter{Introdução}
\lipsum[2]

\chapter{Fundamentação Teórica}
\lipsum[3]

\chapter{Trabalhos Relacionados}
\lipsum[4]

\chapter{Metodologia}
\lipsum[5]

\chapter{Desenvolvimento}
\lipsum[6]

\chapter{Resultados}
\lipsum[7]

\chapter{Conclusão}
\lipsum[8]

\bibliography{referencias}

\appendix
\chapter{Apêndices}
\lipsum[9]

\end{document}
