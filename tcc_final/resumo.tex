O sistema TACO (Task Companion) \cite{taco_repo} consiste em uma aplicação web desenvolvida com React e TypeScript, organizada pelo bundler Vite. Seu propósito é auxiliar a criação e execução de fluxos de atividades interativas, permitindo que usuários montem sequências de passos do tipo texto, pergunta, mídia ou componente customizado. Cada fluxo é salvo localmente por meio da biblioteca Dexie, que abstrai o IndexedDB, garantindo persistência sem necessidade de servidor.

O editor de fluxos possibilita operações de criar, clonar, importar e exportar definições. Na importação, os identificadores de passos são regenerados para evitar conflitos. Um grafo de transições é gerado automaticamente para cada fluxo, servindo de base para análise de caminhos percorridos. O módulo de reprodução (Flow Player) inicia sessões de execução, registra passagem por cada passo. Caso o usuário complete o fluxo, incrementa-se o contador de conclusões e é gravado um log detalhado.

Todos os eventos relevantes são armazenados em coleções específicas, possibilitando estatísticas como taxa de conclusão e tempo médio por passo. Além disso, há suporte a componentes customizados, configuração de cores e logotipo da empresa, lista de auditoria e exportação de dados. A interface utiliza Tailwind CSS e Radix UI para oferecer experiência responsiva e acessível, funcionando também como PWA. Assim, o TACO se apresenta como uma solução completa e local para gerenciamento de procedimentos e acompanhamento de produtividade.

O modo offline garante acesso irrestrito aos fluxos, beneficiando organizações que atuam em locais sem conectividade. Por meio de técnicas de gamificação e feedback contínuo, o sistema incentiva a conclusão das etapas e facilita o aprendizado prático. Os relatórios analíticos permitem otimizar os processos com base em evidências reais, tornando o TACO uma ferramenta versátil para empresas de diferentes portes e segmentos.
